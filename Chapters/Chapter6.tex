% Chapter Template

\chapter{Conclusion} % Main chapter title

\label{Chapter6} % Change X to a consecutive number; for referencing this chapter elsewhere, use \ref{ChapterX}

\lhead{Chapter 6. \emph{Conclusion}} % Change X to a consecutive number; this is for the header on each page - perhaps a shortened title

\section{Findings}
The goal of this thesis was to develop a monitoring system for data center networks using
data plane programmability and debate about the feasibility of a relational model for effectively querying the network. 
In chapter 4, we have 
demonstrated that with a relatively simple set of information about packets leaving switches, a number of quantities can be
derived which are hugely insightful. These quantities were then used in chapter 5 where we saw how a network
administrator could use the visualizations developed to gain an insight into the network. We established network visibility
through peak depth fairness calculations, plots of ingress throughputs and egress throughputs, and an ability to replay the
traffic through the network and see exactly how throughputs across links and queues in switches changed over time.
\section{Future Scope}
The natural next step in this direction is to investigate whether data plane programmability can be combined with
learning methods (techniques of deep learning and machine learning), to create autonomous, self driving networks which
can predict, based on incoming data, as to whether a fault is about to happen in the network and take remedial steps to prevent it.
In cases where the fault couldn't be predicted beforehand, the system can learn from new data and train itself to identify traffic
patterns better. Self driving autonomous data center networks can potentially save a lot of capital for technology companies through
better fault tolerance and fault recovery
as well as give them the ability to
provide better quality of service guarantees to customers.


